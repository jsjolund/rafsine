The purpose of this master thesis is to investigate the usage of the Lattice Boltzmann Method (LBM) of Computational Fluid Dynamics (CFD) for real-time prediction of indoor air flows inside a data center module. Thermal prediction is useful in data centers for evaluating the placement of heat-generating equipment and air conditioning. 

To perform the simulation a program called RAFSINE was used, written by Nicholas Delbosc at the University of Leeds, which implemented LBM on Graphics Processing Units (GPUs) using NVIDIA CUDA. The program used the LBM model called Bhatnagar-Gross-Krook (BGK) on a 3D lattice and had the capability of executing thermal simulations in real-time or faster than real-time. This fast rate of execution means a future application for this simulation could be as a predictive input for automated air conditioning control systems, or for fast generation of training data sets for automatic fault detection systems using machine learning.

In order to use the LBM CFD program even from hardware not equipped with NVIDIA GPUs it was deployed on a remote networked server accessed through Virtual Network Computing (VNC). Since RAFSINE featured interactive OpenGL based 3D visualization of thermal evolution, accessing it through VNC required use of the VirtualGL toolkit which allowed fast streaming of visualization data over the network.

A simulation model was developed describing the geometry, temperatures and air flows of an experimental data center module at RISE SICS North in Lule\r{a}, Sweden, based on measurements and equipment specifications. It was then validated by comparing it with temperatures recorded from sensors mounted in the data center. 

The thermal prediction was found to be accurate on a room-level within $\pm 1$\degree C when measured as the average temperature of the air returning to the cooling units, with a maximum error of $\pm 2$\degree C on an individual basis. Accuracy at the front of the server racks varied depending on the height above the floor, with the lowest points having an average accuracy of $\pm 1$\degree C, while the middle and topmost points had an accuracy of $\pm 2$\degree C and $\pm 4$\degree C respectively. 

While the model had a higher error rate than the $\pm 0.5$\degree C accuracy of the experimental measurements, further improvements could allow it to be used as a testing ground for air conditioning control or automatic fault detection systems.
