In order to know how accurate the \gls{cfd} model of POD 2 described in chapter~\ref{ch:modeling} was, it had to be compared against experimental values using data recorded during the physical experiment.

The temperature measurements on the front and back sides of the server racks were performed using the Microchip temperature sensor MCP9808 which according to specifications had a $\pm 0.5$ \degree C maximum accuracy. Unfortunately, no experimental measurements of air velocity or flow rate were available for the server racks. While the \gls{crac}s had integrated sensors for temperature and mass flow rate (which can be accurately converted to volumetric flow rate using air density at 1 atm pressure), the model and accuracy of these sensors were unknown.

During initial testing of the model a lattice resolution of approximately 27 $lu$ per meter, or 3.75 cm per $lu$, was used based on previous simulation domains created by the original author~\cites[pg.163]{Delbosc}. When discretizing the model this resulted in approximately $1\cdot 10^6$ lattice sites. It was however discovered that the accuracy of this resolution when setting and measuring volumetric flow rates was unacceptably poor, giving a maximum error of around 0.5 m$^3$/s when setting a flow rate of 2.0 m$^3$/s for the boundary conditions on the \gls{crac}s.

Increasing the resolution to 36 $lu$ per meter, or 2.7 cm per $lu$, resulted in more accurate flow rates of less than half of the previous error. The discretized model the had approximately $7\cdot 10^6$ lattice sites, which naturally slowed down the rate at which the model could be simulated relative to real time. Likely, a finer resolution would have given even better results but because of time constraints this was not investigated.

Because of the large number of measurement points the complete record of the thermal simulation along with the experimental values is available in appendix \ref{app:plots}.

\subsubsection{CRACs}

\CountLinesInFile{Plots/P02HDZ01_T_in.csv}

Table~\ref{tab:rms_crac_temps} shows the \gls{rms} of the differences between a series of simulated temperatures and flow rates, $\vec{T}_{sim}$ and $\vec{Q}_{sim}$, and experimental, $\vec{T}_{exp}$ and $\vec{Q}_{exp}$, for a simulation running for $n=\arabic{FileLines}$ minutes. Each sample was averaged over one minute, resulting in $n$ values, so the \gls{rms} could be calculated as
%\begin{equation}
%T_{rms} = \frac{1}{n} \sqrt{ (T_{sim,1}-T_{exp,1})^2+(T_{sim,2}-T_{exp,2})^2+\ldots+(T_{sim,n}-T_{~exp,n})^2}.
%\end{equation}
\begin{equation}
T_{rms} = \sqrt{ \frac{1}{n} \sum\limits_{i=1}^{n} (T_{sim,i}-T_{~exp,i})^2 }
\end{equation}
and similarly for the $Q$ value series. 

%Since the temperature of the air at the \gls{crac} inlets is completely dependent on the \gls{crac} exhaust temperatures and heat increase of the air as it is travels through the server racks, this is the most interesting result regarding the \gls{crac}s. 

\begin{table}[h]
\caption{The \gls{rms} of the difference between simulated and experimental temperatures in \degree C, and volumetric flow rate in m$^3$/s at the inlets of the four \gls{crac} units.}
\begin{center}
	\begin{tabular}{|l|m{1.7cm}|m{1.7cm}|m{1.7cm}|m{1.7cm}|}%
	\hline
	\bfseries CRAC & \bfseries \mbox{Inlet} temp.  & \bfseries \mbox{Exhaust} temp. & \bfseries \mbox{Inlet} flow & \bfseries \mbox{Exhaust} flow
	\csvreader[head to column names]{Plots/P02HDZX_TQ_rms.csv}{}% use head of csv as column names
	{\\\hline\crac&\Tin&\Tout&\Qin&\Qout}% specify your coloumns here
	\\\hline
	\end{tabular}
\end{center}
\label{tab:rms_crac_temps}
\end{table}

As can be seen in table~\ref{tab:rms_crac_temps} the temperature difference at the inlets was on average just over 1~\degree C off from experiments. Of particular note however is the simulated temperature of \gls{crac} number 3, which according to the non-RMS average shown in plot \ref{fig:cracs_mean_temp_bars} was $1.91$~\degree C higher than the experimental value. The reason for this is not clear, but could have to do with the fact that \gls{crac} 1 and 2 had slightly lower volumetric flow rate at the inlets than expected (plots \ref{fig:P02HDZ01_plot} and \ref{fig:P02HDZ02_plot}), likely because of discretization errors in the intake boundary conditions. Since they were lower than expected, they would contribute less to the cooling of the air, leaving temperatures at other \gls{crac}s slightly higher. This is not apparent in the result for the intake at \gls{crac} 4 however, so the reason remains unknown.

The reason that the error is not equal on all \gls{crac} inlets could likely be explained by inaccuracies in the volumetric air flow set at the \gls{crac} boundary conditions, which can also be seen in the table. As mentioned earlier, a higher resolution lattice would probably help mitigate this. Other sources of error could relate to the air turbulence caused by air flows of some of the server racks.

In any case, the results show that the \gls{lbm} model tested was accurate within approximately $\pm 2$~\degree C when the room temperature was measured as the average of the area adjacent to the \gls{crac} air inlets. Figures \ref{fig:P02HDZ01_plot} to \ref{fig:P02HDZ04_plot} in appendix \ref{app:plots} shows the complete record of the \gls{crac} temperatures over time.

\subsubsection{Server Racks}

For the temperatures at the inlets (fronts) and exhausts (backs) of the server racks shown in table~\ref{tab:rms_rack_temps}, the \gls{rms} of the difference between the simulation and experiment was calculated in the same way as those of the \gls{crac}s. The table shows that on average, the temperatures measured at the lowest point on the server racks was accurate within $\pm 1$~\degree C, the middle within $\pm 2$~\degree C and at the top $\pm 4$~\degree C. There are likely multiple sources for this error.

The simulation modeled the fronts of every rack as one uniform boundary condition which was geometrically identical to all others of the same type. In reality, the structure of the racks was more complex. Each rack had a perforated door in front of the servers on which the temperature sensor strips were placed, behind which the vertical positioning of the servers was not uniformly spaced. Some racks were only half full, resulting in a non-uniform air flow through the surface of the rack fronts. This was not taken into account by the model. 

\begin{table}[h]
\caption{The \gls{rms} of the difference between simulated and experimental temperatures in \degree C at different positions on the server racks.}
\begin{center}
	\begin{tabular}{|l|m{1.7cm}|m{1.7cm}|m{1.7cm}|m{1.7cm}|m{1.7cm}|m{1.7cm}|}%
	\hline
	\bfseries Rack & \bfseries Front Bottom & \bfseries Front Mid & \bfseries Front Top & \bfseries Back Bottom & \bfseries Back Mid & \bfseries Back Top
	\csvreader[head to column names]{Plots/P02RX_T_rms.csv}{}% use head of csv as column names
	{\\\hline\rack&\Tinb&\Tinm&\Tint&\Toutb&\Toutm&\Toutt}% specify your coloumns here
	\\\hline
	\end{tabular}
\end{center}
\label{tab:rms_rack_temps}
\end{table}

Another source of error could be in the vertical positioning of the simulation measuring points related to the positioning of the experimental temperature sensor strips. Inaccuracies could also come from the modeling of viscosity (see section \ref{sec:turbulence_modeling}) and thermal expansion coefficient of air (see section \ref{sec:natural_convection}). These variables affect how the heat of air spreads by turbulence and gravity respectively.

Figures \ref{fig:P02R01_plot} to \ref{fig:P02R10_plot} in appendix \ref{app:plots} shows the complete record of the rack temperatures over time. Of particular note are the temperatures of racks P02R01, P02R02 and P02R03, where the fan speed and flow specifications were missing. They are therefore not a good indicator of the accuracy of the thermal model.

As for the temperatures at the backs of the servers, the comparison shows the temperatures being accurate within approximately $\pm 6$~\degree C. Here the same sources of errors apply as for the fronts of the servers, but since the air temperature and velocity is higher and than at the front, the turbulence is more extreme and air currents more erratic. As mentioned in chapter~\ref{ch:modeling}, the average RPM of all servers in a rack was used to calculate the air flow through the rack. A more detailed model could set the air flow individually for each server. As also mentioned in this chapter, the air flow through some of the racks was completely unknown because of missing data for fan speeds. 

Overall, a more accurate model of the positioning and air flow calculation of the servers in the racks, as well as a higher resolution lattice would likely improve these results. 

