%%% API
%%% The glossary entry the acronym links to
\newglossaryentry{apig}{name={API},
    description={An Application Programming Interface~(API) is a particular set
    of rules and specifications that a software program can follow to access and
    make use of the services and resources provided by another particular software
    program that implements that API.
}}
%%% define the acronym and use the see= option
\newglossaryentry{api}{type=\acronymtype, name={API},
description={Application Programming Interface.},
first={Application Programming Interface~(API)\glsadd{apig}}, see=[Glossary:]{apig}}

%%% CPU
\newglossaryentry{cpug}{name={CPU},
    description={A Central Processing Unit~(CPU) is an electronic hardware circuit inside a computer which can perform arithmetic, logic, input/output and flow control operations from computer program instructions.}}
\newglossaryentry{cpu}{type=\acronymtype, name={CPU},
description={Central Processing Unit.},
first={Central Processing Unit~(CPU)\glsadd{cpug}}, see=[Glossary:]{cpug}}

%%% GPU
\newglossaryentry{gpug}{name={GPU},
    description={A Graphics Processing Unit~(GPU) is an electronic hardware circuit specialized for manipulating and drawing computer graphics.}}
\newglossaryentry{gpu}{type=\acronymtype, name={GPU},
description={Graphics Processing Unit.},
first={Graphics Processing Unit~(GPU)\glsadd{gpug}}, see=[Glossary:]{gpug}}

%%% CFD
\newglossaryentry{cfdg}{name={CFD},
    description={Computational Fluid Dynamics~(CFD) is a branch of fluid mechanics which uses numerical analysis and data structures to solve the dynamics of fluid behaviours.}}
\newglossaryentry{cfd}{type=\acronymtype, name={CFD},
description={Computational Fluid Dynamics.},
first={Computational Fluid Dynamics~(CFD)\glsadd{cfdg}}, see=[Glossary:]{cfdg}}

%%% LBM
\newglossaryentry{lbmg}{name={LBM},
    description={The Lattice Boltzmann method~(LBM) is a \gls{cfd} method for fluid~(e.g. liquid, gas and plasma) simulation.}}
\newglossaryentry{lbm}{type=\acronymtype, name={LBM},
description={Lattice Boltzmann method.},
first={Lattice Boltzmann method~(LBM)\glsadd{lbmg}}, see=[Glossary:]{lbmg}}

%%% SSH
\newglossaryentry{sshg}{name={SSH},
    description={Secure Shell~(SSH) is a cryptographic network protocol for secure access and control of remote computer systems.}}
\newglossaryentry{ssh}{type=\acronymtype, name={SSH},
description={Secure Shell.},
first={Secure Shell~(SSH)\glsadd{sshg}}, see=[Glossary:]{sshg}}

%%% VNC
\newglossaryentry{vncg}{name={VNC},
    description={Virtual Network Computing~(VNC) is a graphical desktop sharing system which allows keyboard and mouse events to be transmitted to a remote computer system, while receiving graphical screen updates.}}
\newglossaryentry{vnc}{type=\acronymtype, name={VNC},
description={Virtual Network Computing.},
first={Virtual Network Computing~(VNC)\glsadd{vncg}}, see=[Glossary:]{vncg}}

%%% CUDA
\newglossaryentry{cudag}{name={CUDA},
    description={NVIDIA CUDA~(CUDA) is a parallel computing platform developed by NVIDIA. Its main usage is to perform general purpose data processing on \gls{gpu}s.}}
\newglossaryentry{cuda}{type=\acronymtype, name={CUDA},
description={NVIDIA CUDA.},
first={NVIDIA CUDA~(CUDA)\glsadd{cudag}}, see=[Glossary:]{cudag}}

%%% OpenGL
\newglossaryentry{openglg}{name={OpenGL},
    description={Open Graphics Library~(OpenGL) is a cross-platform \gls{api} for rendering 2D and 3D graphics.}}
\newglossaryentry{opengl}{type=\acronymtype, name={OpenGL},
description={Open Graphics Library.},
first={Open Graphics Library~(OpenGL)\glsadd{openglg}}, see=[Glossary:]{openglg}}

%%% X11
\newglossaryentry{x11g}{name={X11},
    description={The X Window System~(X11) is a graphical windowing system used primarily in \gls{unix}-like operating systems like Linux.}}
\newglossaryentry{x11}{type=\acronymtype, name={X11},
description={X Window System.},
first={X Window System~(X11)\glsadd{x11g}}, see=[Glossary:]{x11g}}

%%% JPEG
\newglossaryentry{jpegg}{name={JPEG},
    description={Joint Photographic Experts Group~(JPEG) is a popular file format for storing compressed bitmap images.}}
\newglossaryentry{jpeg}{type=\acronymtype, name={JPEG},
description={Joint Photographic Experts Group.},
first={Joint Photographic Experts Group~(JPEG)\glsadd{jpegg}}, see=[Glossary:]{jpegg}}

%%% C++
\newglossaryentry{c++g}{name={C++},
    description={The~(C++) programming language is a general purpose multi-paradigm programming language with imperative and object oriented features.}}
\newglossaryentry{c++}{type=\acronymtype, name={C++},
description={Programming language.},
first={C++ programming language~(C++)\glsadd{c++g}}, see=[Glossary:]{c++g}}

%%% LUA
\newglossaryentry{luag}{name={Lua},
    description={The~(Lua) programming language is a lightweight general purpose multi-paradigm programming language with scripting, imperative, functional and object oriented features.}}
\newglossaryentry{lua}{type=\acronymtype, name={Lua},
description={Scripting language.},
first={Lua scripting language~(Lua)\glsadd{luag}}, see=[Glossary:]{luag}}

%%% GLEW
\newglossaryentry{glewg}{name={GLEW},
    description={The \gls{opengl} Extension Wrangler Library~(GLEW) is a cross-platform open-source C/\gls{c++} extension loading library.}}
\newglossaryentry{glew}{type=\acronymtype, name={GLEW},
description={OpenGL Extension Wrangler Library.},
first={OpenGL Extension Wrangler Library~(GLEW)\glsadd{glewg}}, see=[Glossary:]{glewg}}

%%% GLX
\newglossaryentry{glxg}{name={GLX},
    description={The \gls{opengl} extension to \gls{x11}~(GLX) is a software interface which enables applications to access an \gls{opengl} context inside a graphical window provided by \gls{x11}. The GLX libary is usually provided by graphics card manufacturers as part of their drivers.}}
\newglossaryentry{glx}{type=\acronymtype, name={GLX},
description={OpenGL extension to the X Window System.},
first={OpenGL extension to the X Window System~(GLX)\glsadd{glxg}}, see=[Glossary:]{glxg}}

%%% BTE
\newglossaryentry{bteg}{name={BTE},
    description={The Boltzmann transport equation~(BTE) was developed by Ludwig Boltzmann to describe the statistical behavior of a thermodynamic system not in a state of equilibrium.}}
\newglossaryentry{bte}{type=\acronymtype, name={BTE},
description={Boltzmann transport equation.},
first={Boltzmann Transport Equation~(BTE)\glsadd{bteg}}, see=[Glossary:]{bteg}}

%%% UNIX
\newglossaryentry{unixg}{name={UNIX},
    description={UNIX is a family operating systems that derive from the original AT\&T Unix developed during the 1970s. The most popular member is called Linux, followed by BSD.}}
\newglossaryentry{unix}{type=\acronymtype, name={UNIX},
description={A family of operating systems.},
first={UNIX\glsadd{unixg}}, see=[Glossary:]{unixg}}

%%% LibGL
\newglossaryentry{libglg}{name={LibGL},
    description={LibGL implements the \gls{glx} interface. Software based on \gls{opengl} must link to this library to access its entry points. Responsible for dispatching \gls{opengl} library calls to the graphics card driver.}}
\newglossaryentry{libgl}{type=\acronymtype, name={LibGL},
description={Library which implements the \gls{glx} interface.},
first={LibGL\glsadd{libglg}}, see=[Glossary:]{libglg}}

%%% Xlib
\newglossaryentry{xlibg}{name={Xlib},
    description={Xlib is a C/\gls{c++} library which allows programs to interact with \gls{x11} events and commands, such as input devices and graphical window data.}}
\newglossaryentry{xlib}{type=\acronymtype, name={Xlib},
description={C Language X11 Interface.},
first={Xlib\glsadd{xlibg}}, see=[Glossary:]{xlibg}}

%%% DRI
\newglossaryentry{drig}{name={DRI},
    description={Direct Rendering Interface~(DRI) is a framework which allows an \gls{opengl} based application to directly access the graphics hardware~(the \gls{gpu}).}}
\newglossaryentry{dri}{type=\acronymtype, name={DRI},
description={Direct Rendering Interface.},
first={Direct Rendering Interface~(DRI)\glsadd{drig}}, see=[Glossary:]{drig}}

%%% VirtualGL
\newglossaryentry{vglg}{name={VirtualGL},
    description={VirtualGL is an open source toolkit that gives any Unix or Linux remote display software the ability to run \gls{opengl} applications with full 3D hardware acceleration.}}
\newglossaryentry{vgl}{type=\acronymtype, name={VirtualGL},
description={Toolkit for 3D hardware acceleration over remote display software.},
first={VirtualGL\glsadd{vglg}}, see=[Glossary:]{vglg}}

%%% GDB
\newglossaryentry{gdbg}{name={GDB},
    description={The GNU Debugger~(GDB) is an open-source debugger for many programming languages such as \gls{c++}, Fortran, Go, Java etc.}}
\newglossaryentry{gdb}{type=\acronymtype, name={GDB},
description={The GNU Debugger.},
first={The GNU Debugger~(GDB)\glsadd{gdbg}}, see=[Glossary:]{gdbg}}

%%% pthreads
\newglossaryentry{pthreadsg}{name={pthreads},
    description={POSIX Threads (pthreads), is a multi-threading execution model which is implemented in Linux, Mac OS X and Android among others.}}
\newglossaryentry{pthreads}{type=\acronymtype, name={pthreads},
description={POSIX Threads.},
first={POSIX Threads~(pthreads)\glsadd{pthreadsg}}, see=[Glossary:]{pthreadsg}}

%%% FLTK
\newglossaryentry{fltkg}{name={FLTK},
    description={Fast Light Toolkit~(FLTK) is a graphical widget library for \gls{gui}s supporting 3D \gls{opengl} rendering.}}
\newglossaryentry{fltk}{type=\acronymtype, name={FLTK},
description={Fast Light Toolkit.},
first={Fast Light Toolkit~(FLTK)\glsadd{fltkg}}, see=[Glossary:]{fltkg}}

%%% mutex
\newglossaryentry{mutexg}{name={mutex},
    description={Mutual Exclusion~(mutex) is a concept in computer science which acts as locking mechanism allowing only one execution thread at time to manipulate shared memory.}}
\newglossaryentry{mutex}{type=\acronymtype, name={mutex},
description={Mutual Exclusion.},
first={Mutual Exclusion~(mutex)\glsadd{mutexg}}, see=[Glossary:]{mutexg}}

%%% GUI
\newglossaryentry{guig}{name={GUI},
    description={A Graphical User Interface~(GUI) allows the user of a computer program to interact with it using visual indicators and graphical icons instead of text-based user interfaces.}}
\newglossaryentry{gui}{type=\acronymtype, name={GUI},
description={Graphical User Interface.},
first={Graphical User Interface~(GUI)\glsadd{guig}}, see=[Glossary:]{guig}}

%%% PDE
\newglossaryentry{pdeg}{name={PDE},
    description={A Parital Differential Equation~(PDE) is a differential equation with unknown multivariable functions and their partial derivatives.}}
\newglossaryentry{pde}{type=\acronymtype, name={PDE},
description={Parital Differential Equation.},
first={Partial Differential Equations~(PDEs)\glsadd{pdeg}}, see=[Glossary:]{pdeg}}

%%% IBVP
\newglossaryentry{ibvpg}{name={IBVP},
    description={An Initial Boundary Value Problem~(IBVP) is a system of differential equations constrained by initial conditions and boundary conditions unique for a specific problem.}}
\newglossaryentry{ibvp}{type=\acronymtype, name={IBVP},
description={Initial Boundary Value Problem.},
first={Initial Boundary Value Problem~(IBVP)\glsadd{ibvpg}}, see=[Glossary:]{ibvpg}}

%%% BGK
\newglossaryentry{bgkg}{name={BGK},
    description={Bhatnagar--Gross--Krook~(BGK) is a simple model for calculating the collision step in \gls{lbm}.}}
\newglossaryentry{bgk}{type=\acronymtype, name={BGK},
description={Bhatnagar--Gross--Krook.},
first={Bhatnagar--Gross--Krook~(BGK)\glsadd{bgkg}}, see=[Glossary:]{bgkg}}

%%% SM
\newglossaryentry{smg}{name={SM},
    description={Streaming Multiprocessor~(SM) is a \gls{gpu} integrated circuit introduced with the NVIDIA Fermi architecture.}}
\newglossaryentry{sm}{type=\acronymtype, name={SM},
description={Streaming Multiprocessor.},
first={Streaming Multiprocessors~(SM)\glsadd{smg}}, see=[Glossary:]{smg}}

%%% CRAC
\newglossaryentry{cracg}{name={CRAC},
    description={A Computer Room Air Conditioner~(CRAC) is an air cooling device containing heat exchange cooling coils and fans.}}
\newglossaryentry{crac}{type=\acronymtype, name={CRAC},
description={Computer Room Air Conditioner.},
first={Computer Room Air Conditioners~(CRACs)\glsadd{cracg}}, see=[Glossary:]{cracg}}

%%% CSV
\newglossaryentry{csvg}{name={CSV},
    description={Comma-separated values~(CSV) is text table file where the values are separated by commas.}}
\newglossaryentry{csv}{type=\acronymtype, name={CSV},
description={Comma-separated values.},
first={Comma-Separated Values~(CSV)\glsadd{csvg}}, see=[Glossary:]{csvg}}

%%% FVM
\newglossaryentry{fvmg}{name={FVM},
    description={The Finite Volume Method~(FVM) is a numerical method for solving \gls{pde}s using systems of algebraic equations. The problem domain is discretized into smaller volumes yielding a system where the value for each point can be approximated.}}
\newglossaryentry{fvm}{type=\acronymtype, name={FVM},
description={Finite Volume Method.},
first={Finite Volume Method~(FVM)\glsadd{fvmg}}, see=[Glossary:]{fvmg}}

%%% FEM
\newglossaryentry{femg}{name={FEM},
    description={The Finite Element Method~(FEM) is a method for solving \gls{pde}s similar to \gls{fvm}.}}
\newglossaryentry{fem}{type=\acronymtype, name={FEM},
description={Finite Element Method.},
first={Finite Element Method~(FEM)\glsadd{femg}}, see=[Glossary:]{femg}}

%%% FDM
\newglossaryentry{fdmg}{name={FDM},
    description={The Finite Difference Method~(FDM) is a numerical method for solving \gls{pde}s by approximating them with difference equations.}}
\newglossaryentry{fdm}{type=\acronymtype, name={FDM},
description={Finite Difference Method.},
first={Finite Difference Method~(FDM)\glsadd{fdmg}}, see=[Glossary:]{fdmg}}

%%% RISE SICS North
\newglossaryentry{rsng}{name={RISE SICS North},
    description={Research Institutes of Sweden, Swedish Institute of Computer Science, North, (RISE SICS North) is a research institute dedicated to the development of data centers and related technology, located in northern Sweden.}}
\newglossaryentry{rsn}{type=\acronymtype, name={RISE SICS North},
description={Research Institutes of Sweden, Swedish Institute of Computer Science, North.},
first={Research Institutes of Sweden, Swedish Institute of Computer Science, North~(RISE SICS North)\glsadd{rsng}}, see=[Glossary:]{rsng}}

%%% RMS
\newglossaryentry{rmsg}{name={RMS},
    description={The Root Mean Square~(RMS) is a statistical measure defined as the square root of the mean of the squares of a set of numbers.}}
\newglossaryentry{rms}{type=\acronymtype, name={RMS},
description={Root Mean Square.},
first={Root Mean Square~(RMS)\glsadd{rmsg}}, see=[Glossary:]{rmsg}}

%%% LES
\newglossaryentry{lesg}{name={LES},
    description={Large Eddy Simulation~(LES) is a mathematical model for fluid turbulence commonly used in \gls{cfd}.}}
\newglossaryentry{les}{type=\acronymtype, name={LES},
description={Large Eddy Simulation.},
first={Large Eddy Simulation~(LES)\glsadd{lesg}}, see=[Glossary:]{lesg}}

%%% RANS
\newglossaryentry{ransg}{name={RANS},
    description={The Reynolds Averaged Navier-Stokes~(RANS) equations model the motion of fluid flow by time-averaging and are commonly used to model turbulent flows.}}
\newglossaryentry{rans}{type=\acronymtype, name={RANS},
description={Reynolds Averaged Navier-Stokes.},
first={Reynolds Averaged Navier-Stokes~(RANS)\glsadd{ransg}}, see=[Glossary:]{ransg}}
